\hypertarget{index_intro_sec}{}\section{Wstęp...}\label{index_intro_sec}
Tag\-It! to desktopowa aplikacja na systemy Linux, umożliwiająca wygodne, automatyczne tagowanie i nazywanie plików z muzyką, zapewniając przejrzystość w folderach. Dzięki temu można będzie łatwo odnaleźć konkretny utwór, czy też wszystkie utwory danego wykonawcy.\hypertarget{index_install_sec}{}\section{Szczegółowy opis\-:}\label{index_install_sec}
Użytkownik Tag\-It! może wskazać do przetwarzania pojedynczy plik muzyczny, po czym, zależnie od wybranych opcji nastąpi odpowiednie zmodyfikowanie jego nazwy, metadanych oraz ewentualne dodanie okładki albumu, z którego pochodzi. Po wyznaczeniu danego katalogu, aplikacja potrafi również rekurencyjnie przetworzyć znajdujące się w nim pliki, zgodnie z tym jak zostało to opisane wyżej. Ukłonem w stronę użytkownika jest opcja stworzenia i monitorowania muzycznej kolekcji z folderu, który w razie potrzeby zostanie podzielony na adekwatne podfoldery. Wówczas po dodaniu do kolekcji nowego utworu, zostanie on automatycznie nazwany i otagowany. Dzięki temu, pliki po przetworzeniu będą miały jednolity format nazw, odpowiednio zmodyfikowane dodatkowe informacje takie jak\-: gatunek muzyczny, wykonawca, nazwa zespołu, czy album, z którego pochodzą, co zapewni użytkownikowi kontrolę nad jego kolekcją audio. Oczywiście w razie potrzeby ma on również możliwość ręcznej manipulacji metadanymi z poziomu aplikacji.

Tag\-It! obsługuje najpopularniejsze formaty plików takie jak mp3, flac, wav, ogg oraz aac. Ponadto do rozpoznawania plików muzycznych, program łączy się z zewnętrzną bazą online, a więc do poprawnego działania programu wymagane jest połączenie z internetem. Sama baza danych będzie stopniowo rozbudowywana na podstawie różnorakich serwisów muzycznych, stron internetowych, czy też już aktualnie istniejących, ogólnodostępnych, internetowych baz danych przez twórców Tag\-It! oraz odpowiednie web crawlery. Program oferuje również możliwość dodania do bazy metadanych przez użytkownika, w przypadku gdy tagowanie pliku zakończyło się niepowodzeniem. 